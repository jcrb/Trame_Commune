\documentclass[]{article}
\usepackage{lmodern}
\usepackage{amssymb,amsmath}
\usepackage{ifxetex,ifluatex}
\usepackage{fixltx2e} % provides \textsubscript
\ifnum 0\ifxetex 1\fi\ifluatex 1\fi=0 % if pdftex
  \usepackage[T1]{fontenc}
  \usepackage[utf8]{inputenc}
\else % if luatex or xelatex
  \ifxetex
    \usepackage{mathspec}
    \usepackage{xltxtra,xunicode}
  \else
    \usepackage{fontspec}
  \fi
  \defaultfontfeatures{Mapping=tex-text,Scale=MatchLowercase}
  \newcommand{\euro}{€}
\fi
% use upquote if available, for straight quotes in verbatim environments
\IfFileExists{upquote.sty}{\usepackage{upquote}}{}
% use microtype if available
\IfFileExists{microtype.sty}{%
\usepackage{microtype}
\UseMicrotypeSet[protrusion]{basicmath} % disable protrusion for tt fonts
}{}
\usepackage[margin=1in]{geometry}
\usepackage{color}
\usepackage{fancyvrb}
\newcommand{\VerbBar}{|}
\newcommand{\VERB}{\Verb[commandchars=\\\{\}]}
\DefineVerbatimEnvironment{Highlighting}{Verbatim}{commandchars=\\\{\}}
% Add ',fontsize=\small' for more characters per line
\usepackage{framed}
\definecolor{shadecolor}{RGB}{248,248,248}
\newenvironment{Shaded}{\begin{snugshade}}{\end{snugshade}}
\newcommand{\KeywordTok}[1]{\textcolor[rgb]{0.13,0.29,0.53}{\textbf{{#1}}}}
\newcommand{\DataTypeTok}[1]{\textcolor[rgb]{0.13,0.29,0.53}{{#1}}}
\newcommand{\DecValTok}[1]{\textcolor[rgb]{0.00,0.00,0.81}{{#1}}}
\newcommand{\BaseNTok}[1]{\textcolor[rgb]{0.00,0.00,0.81}{{#1}}}
\newcommand{\FloatTok}[1]{\textcolor[rgb]{0.00,0.00,0.81}{{#1}}}
\newcommand{\CharTok}[1]{\textcolor[rgb]{0.31,0.60,0.02}{{#1}}}
\newcommand{\StringTok}[1]{\textcolor[rgb]{0.31,0.60,0.02}{{#1}}}
\newcommand{\CommentTok}[1]{\textcolor[rgb]{0.56,0.35,0.01}{\textit{{#1}}}}
\newcommand{\OtherTok}[1]{\textcolor[rgb]{0.56,0.35,0.01}{{#1}}}
\newcommand{\AlertTok}[1]{\textcolor[rgb]{0.94,0.16,0.16}{{#1}}}
\newcommand{\FunctionTok}[1]{\textcolor[rgb]{0.00,0.00,0.00}{{#1}}}
\newcommand{\RegionMarkerTok}[1]{{#1}}
\newcommand{\ErrorTok}[1]{\textbf{{#1}}}
\newcommand{\NormalTok}[1]{{#1}}
\usepackage{longtable,booktabs}
\usepackage{graphicx}
\makeatletter
\def\maxwidth{\ifdim\Gin@nat@width>\linewidth\linewidth\else\Gin@nat@width\fi}
\def\maxheight{\ifdim\Gin@nat@height>\textheight\textheight\else\Gin@nat@height\fi}
\makeatother
% Scale images if necessary, so that they will not overflow the page
% margins by default, and it is still possible to overwrite the defaults
% using explicit options in \includegraphics[width, height, ...]{}
\setkeys{Gin}{width=\maxwidth,height=\maxheight,keepaspectratio}
\ifxetex
  \usepackage[setpagesize=false, % page size defined by xetex
              unicode=false, % unicode breaks when used with xetex
              xetex]{hyperref}
\else
  \usepackage[unicode=true]{hyperref}
\fi
\hypersetup{breaklinks=true,
            bookmarks=true,
            pdfauthor={RESURAL (JcB)},
            pdftitle={Activité des structures d'urgences : panorama 2014 de la région ALSACE},
            colorlinks=true,
            citecolor=blue,
            urlcolor=blue,
            linkcolor=magenta,
            pdfborder={0 0 0}}
\urlstyle{same}  % don't use monospace font for urls
\setlength{\parindent}{0pt}
\setlength{\parskip}{6pt plus 2pt minus 1pt}
\setlength{\emergencystretch}{3em}  % prevent overfull lines
\setcounter{secnumdepth}{5}

%%% Use protect on footnotes to avoid problems with footnotes in titles
\let\rmarkdownfootnote\footnote%
\def\footnote{\protect\rmarkdownfootnote}

%%% Change title format to be more compact
\usepackage{titling}

% Create subtitle command for use in maketitle
\newcommand{\subtitle}[1]{
  \posttitle{
    \begin{center}\large#1\end{center}
    }
}

\setlength{\droptitle}{-2em}
  \title{Activité des structures d'urgences : panorama 2014 de la région ALSACE}
  \pretitle{\vspace{\droptitle}\centering\huge}
  \posttitle{\par}
  \author{RESURAL (JcB)}
  \preauthor{\centering\large\emph}
  \postauthor{\par}
  \predate{\centering\large\emph}
  \postdate{\par}
  \date{28/01/2015}



\begin{document}

\maketitle


{
\hypersetup{linkcolor=black}
\setcounter{tocdepth}{3}
\tableofcontents
}
Version mse à jour le: \textbf{02/10/2015}

\section{Activité des structures d'urgences : panorama 2014 de la région
ALSACE}\label{activite-des-structures-durgences-panorama-2014-de-la-region-alsace}

Rapport 2014 respectant les préconisations de la FEDORU. Source:
\href{https://docs.google.com/document/d/101LYVqVLeHZnrujfMm3aqBYfbOwx3CPEB3Y-Lbud2Ls/edit}{Trame
commune}

Le document de référence pour le rapport est: \textbf{V4 trame commune
2014 rapport inter région} (xps:
/home/jcb/Documents/Resural/FEDORU/Trame\_Commune/DOC/Trame commune 2014
rapport inter région (V4).docx)

\textbf{NOTE}: certaines informations utiles sont dans
\textbf{RPU\_Doc}.

\section{LE MOT DU PRÉSIDENT DE LA
FEDORU}\label{le-mot-du-president-de-la-fedoru}

La publication du panorama des urgences de la région
\_\_ALSACE\_\_constitue une excellente occasion pour présenter la
fédération des observatoires régionaux des urgences (FEDORU) qui compte
\textbf{RESURAL} parmi ses membres actifs.

La FEDORU a été créée au mois d'octobre 2013. Ses membres sont chargés
dans leur région respective du traitement des données d'urgences ; ce
point commun est le trait d'origine de la FEDORU et donne son empreinte
à l'objet de notre association que je cite ici :

\begin{itemize}
\itemsep1pt\parskip0pt\parsep0pt
\item
  promouvoir les observatoires régionaux des urgences et les structures
  ayant une activité similaire ;
\item
  promouvoir toutes les actions visant à améliorer la connaissance sur
  les soins de premier recours ;
\item
  partager les expertises dans le domaine du recueil, de l'analyse et de
  l'évaluation de la qualité des données relatives à l'activité des
  urgences.
\end{itemize}

Les premières publications de la FEDORU (disponibles sur le site :
\url{http://www.fedoru.fr}) abordent les thèmes techniques suivants :

\begin{itemize}
\itemsep1pt\parskip0pt\parsep0pt
\item
  Recommandations pour la création d'un ORU
\item
  Collecte et usage des RPU
\item
  Hôpital en tension - Synthèse FEDORU
\end{itemize}

Ces documents constituent le socle indispensable à la conduite de
travaux inter-régionaux. Nous pourrons ainsi comparer nos résultats,
harmoniser les indicateurs retenus dans nos publications respectives,
travailler sur des échantillons de données plus importants(inter-région
ou national), mais aussi évaluer l'impact de différentes organisations.

La recherche de consensus et d'échanges entre les différents acteurs
régionaux représentés au sein de la FEDORU s'illustre parfaitement dans
cette publication qui prend le parti de respecter les premières
recommandations sur le traitement des RPU. Le ``panorama des urgences en
région \ldots{}.'', intègre le format d'analyse commun 2015 proposé de
manière collégiale par nos groupes experts et validé par notre conseil
d'administration. Ce socle d'analyse produit par ``la structure
concernée'' sera rapproché des résultats des autres régions et donnera
lieu à une publication commune au cours de l'année 2015. J'adresse au
nom de la FEDORU toutes mes félicitations à l'ensemble de l'équipe de
\textbf{RESURAL} pour la qualité de leurs travaux mais aussi et surtout
à tous les professionnels des services d'urgences de l'\textbf{ALSACE}
pour le fastidieux mais si précieux travail de collecte sur le terrain.

\textbf{Dr G. VIUDES}

\emph{Président de la FEDORU}

\section{Description de l'offre de
soins}\label{description-de-loffre-de-soins}

\subsection{Etablissements d'Alsace ayant une autorisation de structure
d'urgence}\label{etablissements-dalsace-ayant-une-autorisation-de-structure-durgence}

Sites gégraphiques d'accueil des urgences:

\begin{itemize}
\item
  Territoire de santé 1

\begin{verbatim}
- CH de Wissembourg (SU polyvalent + SMUR)
- CH de Haguenau (SU polyvalent + SMUR)
- CH de Saverne  (SU polyvalent + SMUR)
\end{verbatim}
\item
  Territoire de santé 2

\begin{verbatim}
- CHU de Strasbourg
        - NHC (SU adulte)
        - Hôpital de Hautepierre (SU adulte + SU pédiatrique)
        - Pôle logistique (SAMU + SMUR + Hélismur)
- Clinique Sainte Odile (SU polyvalent)
- Clinique Sainte Anne (SU polyvalent)
- Clinique du Diaconat (SU Mains)
\end{verbatim}
\item
  Territoire de santé 3

\begin{verbatim}
- CH de Sélestat (SU polyvalent + SMUR)
- CH de Colmar 
        - Hôpital Pasteur (SU polyvalent + SMUR)
        - Hôpital du parc (SU pédiatrique)
- CH de Guebwiller (SU polyvalent)
\end{verbatim}
\item
  Territoire de santé 4

\begin{verbatim}
- CH de Thann (SU polyvalent)
- CH d'Altkirch (SU polyvalent)
- CH de Mulhouse
        - Hôpital Emile Muller (SU polyvalent + SAMU + SMUR + Hélismur)
        - Hôpital du Hasenrain (SU pédiatrique)
- Clinique des 3 frontières (SU polyvalent)
- Clinique du Diaconat-Fonderie (SU polyvalent)
- Clinique du Diaconat-Roosvelt (SU Mains)
\end{verbatim}
\end{itemize}

En 2014 tous ces établissements ont fourni des RPU sauf le CH THANN.

\subsection{Qualité des données}\label{qualite-des-donnees}

Réalisation d'un diagramme radar présentant l'exhaustivité des items
RPU.

\includegraphics{rapport2014_V4_files/figure-latex/completude-1.pdf}

\includegraphics{rapport2014_V4_files/figure-latex/radar-1.pdf}

Complétude en valeur absolue:

\begin{verbatim}
          FINESS               ID          EXTRACT      CODE POSTAL 
          416733           416733           415731           416733 
         COMMUNE        NAISSANCE             SEXE    DATE D'ENTREE 
          416716           416733           416733           416733 
   MODE D'ENTREE       PROVENANCE        TRANSPORT    TRANSPORT PEC 
          391370           239122           289308           283189 
  DATE DE SORTIE   MODE DE SORTIE      DESTINATION      ORIENTATION 
          374349           338878            82635            72898 
MOTIF DE RECOURS             CCMU               DP 
          270962           339827           245974 
\end{verbatim}

Complétude en pourcentages:

\begin{verbatim}
          FINESS               ID          EXTRACT      CODE POSTAL 
             100              100              100              100 
         COMMUNE        NAISSANCE             SEXE    DATE D'ENTREE 
             100              100              100              100 
   MODE D'ENTREE       PROVENANCE        TRANSPORT    TRANSPORT PEC 
              94               57               69               68 
  DATE DE SORTIE   MODE DE SORTIE      DESTINATION      ORIENTATION 
              90               81              100               88 
MOTIF DE RECOURS             CCMU               DP 
              65               82               60 
\end{verbatim}

\section{Les chiffres clés de l'activité des services
d'urgences}\label{les-chiffres-cles-de-lactivite-des-services-durgences}

Le format des chiffres clés est celui défini par la FEDORU. Il est
commun à toutes les régions membres de la FEDORU.

\subsection{Recueil des données}\label{recueil-des-donnees}

\begin{itemize}
\itemsep1pt\parskip0pt\parsep0pt
\item
  Population alsacienne au 1er janvier 2014: 1 868 773 \href{}{INSEE}
\item
  Nombre de passages dans l'année: 521 129 (données SAE 2014)
\item
  Nombre de passages pour 10.000 habitants: 2 789
\item
  Nombre de RPU déclarés: \textbf{416 733 RPU}
\item
  Nombre de RPU pour 10.000 habitants: 2 230
\item
  Exhaustivité du recueil: \textbf{79.97 \%}
\item
  Moyenne quotidienne de passages: \textbf{1 142 RPU/jour}
\item
  \%(N) d'évolution par rapport à année 2013: \textbf{22 \%}.
\item
  \% d'évolution moyenne sur les 5 dernières années (méthode calcul :
  \emph{pas de données disponibles}.
\item
  Données renseignées (données à partir desquelles tout le reste de
  l'analyse sera effectuée) = Nombre de RPU transmis: 416 733 RPU
\end{itemize}

\subsection{Patients}\label{patients}

\subsubsection{Sexe}\label{sexe}

\begin{itemize}
\itemsep1pt\parskip0pt\parsep0pt
\item
  \%(N) Femme: 47.78 \% (217 617)
\item
  \%(N) Homme: 52.22 \% (199 110)
\item
  Sex ratio: \textbf{1.09}
\item
  Taux de masculinité: 0.52
\end{itemize}

\subsubsection{Age}\label{age}

\begin{itemize}
\item
  age moyen: \textbf{38 ans}.
\item
  age moyen des hommes: 35.9 ans.
\item
  age moyen des femmes: 40.3 ans.
\item
  \% (N) \textless{} 1 an: 15 376 (\textbf{3.69 \%})
\item
  \%(N) \textless{} 15 ans: 103 413 (\textbf{24.82 \%})
\item
  \%(N) \textless{} 18 ans: 119 213 (\textbf{28.61 \%})
\item
  \%(N) \textgreater{}= 75 ans: 57 271 (\textbf{13.74 \%})
\item
  Pyramide des ages:
\end{itemize}

\includegraphics{rapport2014_V4_files/figure-latex/pyramide-1.pdf}

\subsubsection{Taux de recours (définition FEDORU) régional aux
urgences.}\label{taux-de-recours-definition-fedoru-regional-aux-urgences.}

Le taux de recours régional est calculé à partir des données de l'INSEE.

TARRU: \textbf{21.31\%} (ref: population alsacienne 2014)

\subsubsection{Pourcentage de Patients ne venant pas de la région
(étranger
compris)}\label{pourcentage-de-patients-ne-venant-pas-de-la-region-etranger-compris}

Part des non résidents: \textbf{4.43\%} (N = 18 467)

\subsection{ARRIVÉE}\label{arrivee}

\subsubsection{Horaires de passage}\label{horaires-de-passage}

\includegraphics{rapport2014_V4_files/figure-latex/horaires-1.pdf}
\includegraphics{rapport2014_V4_files/figure-latex/horaires-2.pdf}

\begin{itemize}
\item
  Passages de nuit (20h - 8h): \textbf{27.7 \%} (N = 115 418)
\item
  Passages en nuit profonde (0h - 8h): \textbf{10.38 \%} (N = 43 271)
\item
  Passages en horaire de PDSA: \textbf{45.22 \%} N = 188 454 (Remarque:
  ne tient pas compte des jours fériés survenant en semaine)
\end{itemize}

\subsubsection{Variations saisonnières}\label{variations-saisonnieres}

Variation du nombre de RPU entre les mois d'été (juillet-août) et les
autres mois de l'année: \textbf{-5.82 \%}.

\subsubsection{Moyens d'arrivée}\label{moyens-darrivee}

\begin{itemize}
\itemsep1pt\parskip0pt\parsep0pt
\item
  \%(N) d'arrivée personnel: \textbf{72.16 \%} (N = 208 771)
\item
  \%(N) d'arrivée SMUR: \textbf{0.93 \%} (N = 2 702)
\item
  \%(N) d'arrivée VSAB: \textbf{10.35 \%} (N = 29 954)
\item
  \%(N) d'arrivée Ambulance: \textbf{15.94 \%} (N = 46 112)
\end{itemize}

NB : commentaire possible pour expliquer que la somme des 4 pourcentages
ci dessus ne fait pas 100 \%

\subsubsection{Gravité (CCMU)}\label{gravite-ccmu}

\begin{itemize}
\itemsep1pt\parskip0pt\parsep0pt
\item
  nombre de CCMU renseignés: 339 827.
\item
  \%(N) CCMU 1: \textbf{15.21\%} (n = 51 682)
\item
  \%(N) CCMU 1 et 2: \textbf{84.45\%} (n = 286 979)
\item
  \%(N) CCMU 4 et 5: \textbf{1.28\%} (n = 4 341)
\end{itemize}

Exhaustivité CCMU :

\begin{itemize}
\itemsep1pt\parskip0pt\parsep0pt
\item
  Nombre de RPU 2014 hors orientation = FUGUE, PSA et REO ayant un
  élément transmis pour la CCMU: \textbf{335 889}.
\end{itemize}

\subsubsection{Diagnostic principal}\label{diagnostic-principal}

Remarque: les chiffres sont dans le document
\emph{Codes\_regroupement\_ORUMIP} =\textgreater{} à rajouter.

\begin{itemize}
\item
  \% Médico-chirurgical: \textbf{55.35 \%} (136 816)
\item
  \% Traumatologique: \textbf{37.18 \%} (91 907)
\item
  \% Psychiatrique: \textbf{2.5 \%} (6 185)
\item
  \% Toxicologique: \textbf{1.96 \%} (4 847)
\item
  \% Autres recours: \textbf{3.01 \%} (7 441)
\item
  \% Médico-chirurgical: 55.35 \%

\begin{verbatim}
- dont :

        - % cardio vasculaire:
        - % neuro:
        - % digestif:
        - % respiratoire:
\end{verbatim}
\item
  \% Traumatologique: 37.18 \%
\item
  \% Psychiatrique: 2.5 \%
\item
  \% Toxicologique: 1.96 \%
\item
  \% Autres recours: 3.01 \%
\end{itemize}

\subsubsection{Durées de passage}\label{durees-de-passage}

\includegraphics{rapport2014_V4_files/figure-latex/passages-1.pdf}

\begin{itemize}
\item
  Nombre de RPU dont la durée de passage est comprise entre 0h et 72h:
  \textbf{338 722}
\item
  durée moyenne de passage \textbf{160 mn} (2h40).
\item
  écart-type: 173.22 mn (2h53).
\item
  médiane: \textbf{113 mn} (1h53).
\item
  nombre de prises en charge \textgreater{} 4 heures: 63 101
  (\textbf{18.63 \%}).
\item
  nombre de prises en charge inférieures ou égales à 4 heures: 275 621
  (\textbf{81.37 \%}).
\item
  Lors d'une hospitalisation post-urgences (hospitalisation = mutation +
  transfert)

  \begin{itemize}
  \itemsep1pt\parskip0pt\parsep0pt
  \item
    moyenne durée de passage en cas d'hospitalisation: \textbf{238.69
    mn}.
  \item
    médiane durée de passage en cas d'hospitalisation: \textbf{198 mn}.
  \end{itemize}
\item
  Lors d'un retour au domicile

  \begin{itemize}
  \itemsep1pt\parskip0pt\parsep0pt
  \item
    moyenne durée de passage en cas de retour à domicile: \textbf{145.66
    mn}.
  \item
    médiane durée de passage en cas de retour à domicile: \textbf{103
    mn}.
  \end{itemize}
\end{itemize}

(source: temps de passages.Rmd)

\subsubsection{Mode de sortie}\label{mode-de-sortie}

\begin{itemize}
\itemsep1pt\parskip0pt\parsep0pt
\item
  \% (N) de retour à domicile: \textbf{75.5 \%} (N = 255 852)
\item
  \% (N) Hospitalisation: \textbf{24.5 \%} (N = 83 024)
\item
  \% (N) Mutation: \textbf{22.72 \%} (N = 76 999)
\item
  \% (N) Transfert: \textbf{1.78 \%} (N = 6 025)
\item
  Nb de RPU 2014 avec mode de sortie = 6 ou 7 (hospitalisation) avec un
  élément transmis pour la destination: \textbf{82635}
\item
  Nb de RPU 2014 avec mode de sortie = 6 ou 7 avec un élement transmis
  pour l'orientation: \textbf{72898}
\end{itemize}

\section{Les chiffres clés de l'activité des
SAMU}\label{les-chiffres-cles-de-lactivite-des-samu}

(à partir des données SRVA ``officielles'')

\begin{itemize}
\itemsep1pt\parskip0pt\parsep0pt
\item
  Nombre de dossiers de régulation médicale (DRM): 480303
\item
  Nombre de SMUR : 25 321

  \begin{itemize}
  \itemsep1pt\parskip0pt\parsep0pt
  \item
    dont primaires: 19 714
  \end{itemize}
\item
  Nombre d'ambulances privées à la demande du SAMU:
  \texttt{format.n(r assu)}
\end{itemize}

\subsection{Organisation}\label{organisation}

\subsubsection{Nombre de colonnes SMUR
terrestres:}\label{nombre-de-colonnes-smur-terrestres}

\begin{longtable}[c]{@{}lll@{}}
\toprule
SMUR & Jour & Nuit\tabularnewline
\midrule
\endhead
Wissembourg & 1 & 1\tabularnewline
Haguenau & 1 & 1\tabularnewline
Saverne & 1 & 1\tabularnewline
Strasbourg & 4 & 3\tabularnewline
Sélestat & 1 & 1\tabularnewline
Colmar & 2 & 2\tabularnewline
Mulhouse & 2 & 2\tabularnewline
\bottomrule
\end{longtable}

\subsubsection{Nombre de SMUR
héliportés:}\label{nombre-de-smur-heliportes}

\begin{longtable}[c]{@{}llll@{}}
\toprule
SMUR & Jour & Nuit & Remarques\tabularnewline
\midrule
\endhead
Strasbourg & 1 & 1 & convention avec la sécurité civile\tabularnewline
Mulhouse & 1 & 1 & colonne mutualisée avec le SMUR
terrestre\tabularnewline
\bottomrule
\end{longtable}

\subsubsection{Nombre de SMUR
pédiatriques:}\label{nombre-de-smur-pediatriques}

\begin{longtable}[c]{@{}lll@{}}
\toprule
SMUR & Jour & Nuit\tabularnewline
\midrule
\endhead
Strasbourg (HTP) & 1 & 1\tabularnewline
\bottomrule
\end{longtable}

\subsubsection{SAMU}\label{samu}

\begin{itemize}
\itemsep1pt\parskip0pt\parsep0pt
\item
  Nombre de SAMU: 2
\item
  Nombre de SAMU par bassin populationnel (pour 100 000): 0.11
\item
  Nombre de dossier de régulation médicale pour 100.000 h: 25 702
\item
  Nombre de lignes SMUR par bassin populationnel (pour 100 000): 0.7
\item
  Nombre de SU géographiques par bassin populationnel (pour 100 000):
  1.02
\end{itemize}

\section{Les chiffres clés de l'activité pédiatrique des services
d'urgences (moins de 18
ans)}\label{les-chiffres-cles-de-lactivite-pediatrique-des-services-durgences-moins-de-18-ans}

\subsection{Recueil des données}\label{recueil-des-donnees-1}

\begin{itemize}
\itemsep1pt\parskip0pt\parsep0pt
\item
  Nombre de passages dans l'année: 119 213
\item
  Moyenne quotidienne de passage: 327 passages/j
\item
  Taux d'urgences pédiatriques {[}(Nb RPU Pédia/ Nb RPU global)x100{]}:
  29 \%
\item
  TODO: \% d'évolution par rapport à l'année N-1(données SAE pour ceux
  qui n'ont pas d'historique RPU fiable et permettant la comparaison,
  préciser l'origine des données)
\end{itemize}

\subsection{Patients}\label{patients-1}

\subsubsection{Répartition par tranches
d'âge}\label{repartition-par-tranches-dage}

\begin{table}[ht]
\centering
\begin{tabular}{rr}
  \hline
 & ped \\ 
  \hline
$<$28j & 1 791 \\ 
  28j-1an[ & 13 554 \\ 
  1-5ans[ & 36 287 \\ 
  5-10ans[ & 24 738 \\ 
  10-15ans[ & 27 012 \\ 
  15-18ans[ & 15 800 \\ 
   \hline
\end{tabular}
\caption{Nombre de RPU pédiatriques en 2014 par grandes classes d'âge} 
\end{table}

\includegraphics{rapport2014_V4_files/figure-latex/unnamed-chunk-9-1.pdf}

\subsubsection{Pyramide des âges}\label{pyramide-des-ages}

\includegraphics{rapport2014_V4_files/figure-latex/unnamed-chunk-10-1.pdf}

\begin{itemize}
\itemsep1pt\parskip0pt\parsep0pt
\item
  nombre de garçons: 65 619
\item
  nombre de filles: 53 590
\item
  Sex ratio: 1.22
\item
  Pyramide des âges (âge par année, borne supérieure toujours exclue)
\item
  Par sous classes d'âge:
\end{itemize}

\subsubsection{mode de transport
pédiatrique}\label{mode-de-transport-pediatrique}

nombre de RPU pédiatriques avec un moyen de transport renseigné: 77 690
(p = 65.17 \%)

\begin{table}[ht]
\centering
\begin{tabular}{rrr}
  \hline
 & Fréq. & \% \\ 
  \hline
AMBU & 2 024,00 & 2,61 \\ 
  FO & 81,00 & 0,10 \\ 
  HELI & 17,00 & 0,02 \\ 
  PERSO & 71 809,00 & 92,43 \\ 
  SMUR & 706,00 & 0,91 \\ 
  VSAB & 3 053,00 & 3,93 \\ 
   \hline
\end{tabular}
\caption{Modes de transports pédiatriques} 
\end{table}

\subsubsection{Gravité des RPU
pédiatriques}\label{gravite-des-rpu-pediatriques}

nombre de CCMU pédiatriques renseignés: 94 706 (p = 79.44 \%)

\begin{table}[ht]
\centering
\begin{tabular}{rrr}
  \hline
 & Fréq. & \% \\ 
  \hline
1 & 24 353,00 & 25,71 \\ 
  2 & 63 361,00 & 66,90 \\ 
  3 & 6 678,00 & 7,05 \\ 
  4 & 211,00 & 0,22 \\ 
  5 & 23,00 & 0,02 \\ 
  D & 2,00 & 0,00 \\ 
  P & 78,00 & 0,08 \\ 
   \hline
\end{tabular}
\caption{Gravité des RPU pédiatriques en 2014.} 
\end{table}

\begin{itemize}
\itemsep1pt\parskip0pt\parsep0pt
\item
  \% Médico-chirurgical: 51.59 \% - dont : - \% cardio vasculaire: - \%
  neuro: - \% digestif: - \% respiratoire:
\item
  \% Traumatologique: 44.27 \%
\item
  \% Psychiatrique: 0.94 \%
\item
  \% Toxicologique: 0.71 \%
\item
  \% Autres recours: 2.5 \%
\end{itemize}

\subsubsection{horaires de passages
pédiatriques}\label{horaires-de-passages-pediatriques}

\begin{itemize}
\itemsep1pt\parskip0pt\parsep0pt
\item
  nombre de passages la nuit: 32 677,00, 0,27 (p = 27.41, 0 \%)
\item
  nombre de passages en nuit profonde: 8 660,000, 0,073 (p = 7.26, 0 \%)
\end{itemize}

\subsubsection{Durée de passage}\label{duree-de-passage}

\begin{itemize}
\item
  Nombre de RPU avec une heure de sortie conforme ({]}0-72h{[}: 106 660
\item
  Durée moyenne de passage (en min): 120.76 mn
\item
  Durée médiane de passage (en min): 86 mn
\item
  Nombre de RPU dont la durée de passage est inférieure à 4h: 99 128
\item
  Nombre de RPU avec une heure de sortie conforme ({]}0-72h{[} lors
  d'une hospitalisation post-urgences: 9 537
\item
  Nombre de RPU avec une heure de sortie conforme ({]}0-72h{[} lors d'un
  retour au domicile: 83 458
\item
  Nombre de RPU dont la durée de passage est inférieure à 4h lors d'une
  hospitalisation post-urgences: 8 532
\item
  Nombre de RPU dont la durée de passage est inférieure à 4h lors d'un
  retour au domicile: 90 595
\item
  Nombre de RPU avec un mode de sortie renseigné: 96 860
\item
  Nombre de mutation interne: 11 996
\item
  Nombre de transfert externe: 556
\item
  nombre de retours à domicile: 84 307
\end{itemize}

\section{Les chiffres clés de l'activité gériatrique des services
d'urgences (75 ans et
plus)}\label{les-chiffres-cles-de-lactivite-geriatrique-des-services-durgences-75-ans-et-plus}

En 2014, l'Alsace recense 155 281 personnes de 75 ans ou plus.

\subsection{Recueil des données}\label{recueil-des-donnees-2}

\begin{itemize}
\itemsep1pt\parskip0pt\parsep0pt
\item
  Nombre de passages dans l'année: \textbf{57 271}
\item
  Taux de recours aux SU de la population gériatrique: \textbf{36.88 \%}
\item
  Moyenne quotidienne de passage: \textbf{157 passages/j}
\item
  Taux d'urgences gériatriques {[}(Nb RPU Géria/ Nb RPU global)x100{]}:
  \textbf{13.74 \%}
\item
  \% d'évolution par rapport à l'année N-1: \emph{données non fiables}.
\end{itemize}

\subsection{Patients}\label{patients-2}

\includegraphics{rapport2014_V4_files/figure-latex/sexe75-1.pdf}

\begin{itemize}
\item
  Nombre d'hommes: 22 665
\item
  Nombre de femmes: 34 605
\item
  Sex ratio: 0.65
\item
  Pyramide des âges (âge par année, borne supérieure toujours exclue)
\item
  Par sous classes d'âge (voir \ref{table.synthese.75}):

  \begin{itemize}
  \itemsep1pt\parskip0pt\parsep0pt
  \item
    85 ans ou moins: 33 399
  \item
    plus de 85 ans: 23 872
  \end{itemize}
\end{itemize}

\begin{table}[ht]
\centering
\begin{tabular}{rrrrr}
  \hline
 & effectif & moyenne par jour  & médiane par jour & sex ratio \\ 
  \hline
75-84 ans & 33 399 & 91,50 &  91 & 0,82 \\ 
  85 ans et plus & 23 872 & 65,40 &  66 & 0,47 \\ 
   \hline
\end{tabular}
\caption{Caractéristiques des patients de 75 ans et plus, répartis en deux classes d'âge} 
\end{table}

\subsubsection{Pyramides des ages}\label{pyramides-des-ages}

\includegraphics{rapport2014_V4_files/figure-latex/unnamed-chunk-13-1.pdf}

\subsection{ARRIVÉE}\label{arrivee-1}

\subsubsection{Horaires de passage}\label{horaires-de-passage-1}

\begin{itemize}
\itemsep1pt\parskip0pt\parsep0pt
\item
  Nb de RPU avec date/heure d'entrée renseignés: \textbf{57 271}
\item
  \% passages la nuit: \textbf{22.38 \%} (N = 12 815)
\item
  \% passages en horaire de PDS: \textbf{38.12 \%} (N = 21 830)
\end{itemize}

\subsubsection{Moyens de transport}\label{moyens-de-transport}

\begin{itemize}
\item
  nombre de moyens de transport: 57 271
\item
  nombre de moyens de transport renseignés: 40 878
\item
  nombre de moyens personnels: 11 962
\item
  nombre de SMUR: 698
\item
  nombre de VSAV: 6 797
\item
  nombre d'ambulances privées: 21 370
\item
  \% d'arrivées Moyen perso: \_\textbf{0.29 \%} (N = 11 962)
\item
  \% d'arrivées SMUR: \textbf{0.17 \%} (N = 6 797)
\item
  \% d'arrivées VSAV: \textbf{0.17 \%} (N = 6 797)
\item
  \% d'arrivées ambulance privée: \textbf{0.52 \%} (N = 21 370)
\item
  \% réponses manquantes: \textbf{28.62 \%}
\end{itemize}

NB : commentaire possible pour expliquer que la somme des 4 pourcentages
ci dessus ne fait pas 100 \%

\subsubsection{Gravité}\label{gravite}

\begin{itemize}
\itemsep1pt\parskip0pt\parsep0pt
\item
  Nombre de RPU avec une CCMU renseignée: 47 408
\item
  \% CCMU 1: \textbf{4.32 \%} (N = 2 472)
\item
  \% CCMU 4 et 5: \textbf{3.14 \%} (N = 1 797)
\end{itemize}

\subsubsection{Diagnostic principal}\label{diagnostic-principal-1}

\begin{itemize}
\item
  \% Médico-chirurgical: 71.51 \%

\begin{verbatim}
- dont :
        - % cardio vasculaire:
        - % neuro:
        - % digestif:
        - % respiratoire:
\end{verbatim}
\item
  \% Traumatologique: 25.03 \%
\item
  \% Psychiatrique: 1.23 \%
\item
  \% Toxicologique: 0.57 \%
\item
  \% Autres recours: 1.66 \%
\end{itemize}

\subsubsection{DURÉE}\label{duree}

\begin{verbatim}
##               Mutation Transfert Domicile
## Durée moyenne      219       316      215
## Durée médiane      200       248      174
\end{verbatim}

\begin{verbatim}
## 
##  Welch Two Sample t-test
## 
## data:  passages75$duree by passages75$DEVENIR
## t = -5, df = 40000, p-value = 0.000002
## alternative hypothesis: true difference in means is not equal to 0
## 95 percent confidence interval:
##  -12.7  -5.3
## sample estimates:
## mean in group Domicile     mean in group Hosp 
##                    215                    224
\end{verbatim}

\begin{verbatim}
## [1] 0.0000016
\end{verbatim}

\includegraphics{rapport2014_V4_files/figure-latex/duree_passage_75-1.pdf}

\begin{itemize}
\itemsep1pt\parskip0pt\parsep0pt
\item
  Durée moyenne de passage (HORS UHCD) : 220 minutes
\item
  Durée médiane de passage (HORS UHCD) : 190 minutes
\item
  \% de passages de moins de 4h : 61.22 \%
\item
  lors d'une hospitalisation post-urgences (hospitalisation = mutation +
  transfert): 223.7 minutes.
\item
  lors d'un retour au domicile: 214.71 minutes.
\end{itemize}

\paragraph{Nouveau}\label{nouveau}

\begin{itemize}
\item
  Nombre de RPU avec une heure de sortie conforme ({]}0-72h{[}: 37 603
\item
  Durée moyenne de passage (en min): 246.62 mn
\item
  Durée médiane de passage (en min): 210 mn
\item
  Nombre de RPU dont la durée de passage est inférieure à 4h: 21 716
\item
  Nombre de RPU avec une heure de sortie conforme ({]}0-72h{[} lors
  d'une hospitalisation post-urgences: 17 066
\item
  Nombre de RPU avec une heure de sortie conforme ({]}0-72h{[} lors d'un
  retour au domicile: 17 307
\item
  Nombre de RPU dont la durée de passage est inférieure à 4h lors d'une
  hospitalisation post-urgences: 8 109
\item
  Nombre de RPU dont la durée de passage est inférieure à 4h lors d'un
  retour au domicile: 13 607
\end{itemize}

\subsubsection{MODE DE SORTIE}\label{mode-de-sortie-1}

\includegraphics{rapport2014_V4_files/figure-latex/sortie75-1.pdf}

\begin{verbatim}
## pop75$MODE_SORTIE : 
##           Frequency   %(NA+)   %(NA-)
## Mutation      27196     47.5     58.1
## Domicile      18131     31.7     38.7
## NA's          10430     18.2      0.0
## Transfert      1514      2.6      3.2
## NA                0      0.0      0.0
## Décès             0      0.0      0.0
##                   0      0.0      0.0
##   Total       57271    100.0    100.0
\end{verbatim}

\begin{itemize}
\itemsep1pt\parskip0pt\parsep0pt
\item
  \% d'hospitalisation: 50.13 \% (N = 28 710)

  \begin{itemize}
  \itemsep1pt\parskip0pt\parsep0pt
  \item
    \% de mutation:47.49 \% (N = 27 196)
  \item
    \% de transfert:2.64 \% (N = 1 514)
  \end{itemize}
\item
  \% de retour à domicile:31.66 \% (N = 18 131)
\end{itemize}

\paragraph{rapport régional}\label{rapport-regional}

\begin{itemize}
\itemsep1pt\parskip0pt\parsep0pt
\item
  Nombre de RPU avec un mode de sortie renseigné: 46 841
\item
  Nombre de mutation interne: 27 196
\item
  Nombre de transfert externe: 1 514
\item
  nombre de retours à domicile: 18 131
\end{itemize}

\section{Les chiffres clés de l'activité AVC des services
d'urgences}\label{les-chiffres-cles-de-lactivite-avc-des-services-durgences}

\subsection{Recueil des données}\label{recueil-des-donnees-3}

\begin{itemize}
\itemsep1pt\parskip0pt\parsep0pt
\item
  Nombre d'AVC dans l'année (+ rappeler le pourcentage d'exhaustivité du
  DP par rapport au nombre de RPU): \textbf{2 949}
\item
  Moyenne quotidienne d'AVC: \textbf{8,1 AVC/j}
\item
  \% d'AVC dans l'activité globale: \textbf{1.19 \%}
\end{itemize}

\subsection{Répartition des AVC}\label{repartition-des-avc}

Exemple d'utilisation de la méthode \emph{hist} appliquée aux objets
date-time:

\begin{itemize}
\itemsep1pt\parskip0pt\parsep0pt
\item
  \emph{x} = as.Date(AVC\$ENTREE)
\item
  \emph{breaks} est obligatoire: ``days'', ``weeks'', ``months'',
  ``quarters'', ``years'', ``secs'', ``mins'', ``hours''. Utiliser
  \emph{start.on.monday = TRUE} si \emph{breaks = ``weeks''}.
\item
  \emph{freq} = TRUE (défaut FALSE) pour afficher les fréquences
\item
  \emph{format} permet de coisir l'affichage de la date sur l'axe des x
  \href{https://stat.ethz.ch/R-manual/R-devel/library/base/html/strptime.html}{voir}.
\end{itemize}

\includegraphics{rapport2014_V4_files/figure-latex/hist_avc-1.pdf}

\subsection{Patients}\label{patients-3}

\begin{verbatim}
## c.age
##     [0,5)    [5,10)   [10,15)   [15,20)   [20,25)   [25,30)   [30,35) 
##        10         2         4        10        19        26        26 
##   [35,40)   [40,45)   [45,50)   [50,55)   [55,60)   [60,65)   [65,70) 
##        33        68        97       150       166       235       284 
##   [70,75)   [75,80)   [80,85)   [85,90)   [90,95)  [95,100) [100,105) 
##       295       393       496       408       193        28         4 
## [105,110) [110,115) [115,120) 
##         1         1         0
\end{verbatim}

\includegraphics{rapport2014_V4_files/figure-latex/Patients-1.pdf}

\begin{verbatim}
## c.age
##    [0,1)    [1,5)   [5,10)  [10,15)  [15,18)  [18,30)  [30,45)  [45,65) 
##        3        7        2        4        4       51      127      648 
##  [65,75)  [75,85) [85,120) 
##      579      889      635
\end{verbatim}

\includegraphics{rapport2014_V4_files/figure-latex/Patients-2.pdf}

\begin{itemize}
\itemsep1pt\parskip0pt\parsep0pt
\item
  Sex ratio: 0.95
\item
  Age moyen: 71.44 ans
\item
  Nombre d'AVC par sous classe d'âge (GT1):

  \begin{itemize}
  \itemsep1pt\parskip0pt\parsep0pt
  \item
    85 ans ou moins: 2 404 (81.52 \%)
  \item
    plus de 85 ans: 545 (18.48 \%)
  \end{itemize}
\end{itemize}

\subsection{ARRIVÉE}\label{arrivee-2}

\begin{itemize}
\itemsep1pt\parskip0pt\parsep0pt
\item
  Nombre d'AVC et \% par tranche d'heure GT1 (matinée, début d'après
  midi, fin d'après midi, soirée, nuit profonde)
\end{itemize}

\begin{table}[ht]
\centering
\begin{tabular}{rlllll}
  \hline
 & nuit profonde & matinée & début après-midi & fin après-midi & soirée \\ 
  \hline
Heures & [0,8) & [8,12) & [12,16) & [16,20) & [20,24) \\ 
  Nombre & 272 & 900 & 865 & 619 & 293 \\ 
  \% & 9.22 & 30.52 & 29.33 & 20.99 & 9.94 \\ 
   \hline
\end{tabular}
\caption{Arrivées des AVC} 
\label{avc.arrive}
\end{table}

\includegraphics{rapport2014_V4_files/figure-latex/avc_periode-1.pdf}

\begin{itemize}
\item
  \% AVC le matin: 30.5 \%.
\item
  \% AVC en début d'après-midi: 29.3 \%.
\item
  \% AVC en fin d'après-midi: 21 \%.
\item
  \% AVC en soirée: 9.9 \%.
\item
  \% AVC le nuit profonde: 9.2 \%.
\item
  Nombre de passages AVC urgences, déclinaison par département,
  établissement, année N

  \begin{table}[ht]
  \centering
  \begin{tabular}{rrr}
    \hline
   & Nombre d'AVC & \% des diagnostics \\ 
    \hline
  3Fr & 63.00 & 0.57 \\ 
    Alk & 30.00 & 0.43 \\ 
    Ane &  &  \\ 
    Col & 741.00 & 1.25 \\ 
    Dia &  &  \\ 
    Dts &  &  \\ 
    Geb & 30.00 & 0.19 \\ 
    Hag & 500.00 & 1.48 \\ 
    Hus & 580.00 & 2.06 \\ 
    Mul & 682.00 & 1.45 \\ 
    Odi &  &  \\ 
    Ros &  &  \\ 
    Sav &  &  \\ 
    Sel & 238.00 & 0.86 \\ 
    Wis & 85.00 & 0.78 \\ 
     \hline
  \end{tabular}
  \caption{Nombre d'AVC par établissement et pourcentage des diagnostics d'AVC parmis l'ensemble des diagnostics principaux évoqués par établissement de santé.} 
  \label{avcParES}
  \end{table}
\item
  \% passages en horaire de PDS

  \begin{longtable}[c]{@{}rlll@{}}
  \toprule
  PDS & S PDS & WE NPD & S\tabularnewline
  \midrule
  \endhead
  Nombre AVC & 403 & 656 & 1890\tabularnewline
  \% AVC & 14 & 22 & 64\tabularnewline
  \bottomrule
  \end{longtable}
\end{itemize}

PDSS = horaires de PDS en semaine, PDSWE = horaires de PDS le WE, NPDS =
hors horaire de PDS.

\begin{itemize}
\itemsep1pt\parskip0pt\parsep0pt
\item
  nombre d'AVC aux horaires de PDS en semaine: 13.67 \%
\item
  nombre d'AVC aux horaires de PDS de week-end:22.24 \%
\item
  nombre d'AVC en dehors des horaires de PDS:64.09 \%
\item
  Nombre de RPU avec diag AVC avec date et heure d'entrées renseignées:
  2 949
\end{itemize}

\subsection{Mode d'arrivée aux
urgences}\label{mode-darrivee-aux-urgences}

\begin{itemize}
\itemsep1pt\parskip0pt\parsep0pt
\item
  Nombre de RPU avec moyens de transport précisé: 2 395
\item
  \% d'arrivées Moyen perso: 21.57\%
\item
  \% d'arrivées SMUR: 1.97\%
\item
  \% d'arrivées VSAV: 17.87\%
\item
  \% d'arrivées ambulance privée: 39.13\% NB : commentaire possible pour
  expliquer que la somme des 4 pourcentages ci dessus ne fait pas 100 \%
\end{itemize}

\subsection{Diagnostic principal}\label{diagnostic-principal-2}

\begin{itemize}
\itemsep1pt\parskip0pt\parsep0pt
\item
  Nombre d'AVC ischémiques et \%: 1 021 (34.62 \%)
\item
  Nombre d'AVC hémorragiques et \%: 442 (14.99 \%)
\item
  Nombre d'AIT et \%: 806 (27.33 \%)
\item
  Nombre de codes ``symptomatiques'' (hémiplégie, aphasie, amaurose,
  etc\ldots{}) et \%: 680 (23.06 \%)
\end{itemize}

NB : se référer à l'annexe 4 pour les regroupements.

\subsection{DURÉE}\label{duree-1}

Voir ligne 333

Voir les routines de RPU\_2014/Analyse/Temps\_passage/passage.R et
notamment \textbf{temps de passage}.

\begin{itemize}
\item
  Nombre de RPU avec une heure de sortie conforme ({]}0-72h{[}: 1 899
\item
  Durée moyenne de passage des Patients PEC pour AVC (en min): 290
\item
  Durée médiane de passage des Patients PEC pour AVC (en min): 255
\item
  Nombre de RPU ac diag AVC dont la durée de passage est inférieure à
  4h: 878
\item
  Durée de passage (HORS UHCD) année N: moyenne \textbf{249.8} minutes,
  et médiane \textbf{228} minutes.
\item
  \% de passages de moins de 4h 0.92
\end{itemize}

\subsection{MODE DE SORTIE}\label{mode-de-sortie-2}

\includegraphics{rapport2014_V4_files/figure-latex/avc_mode_sortie-1.pdf}

\begin{itemize}
\itemsep1pt\parskip0pt\parsep0pt
\item
  Nombre de RPU ac diag. AVC avec un mode de sortie renseigné: 2626
\item
  \% d'hospitalisation: 87.3 \% (N = 2292)
\item
  \% de mutation: 81.3 \% (N = 2134)
\item
  \% de transfert: 6 \% (N = 158)
\item
  \% de retour à domicile: 12.7 \% (N = 334)
\end{itemize}

\subsection{Orientation}\label{orientation}

\begin{itemize}
\itemsep1pt\parskip0pt\parsep0pt
\item
  Répartition par orientation en pourcentage, année N
\end{itemize}

\% Table created by stargazer v.5.2 by Marek Hlavac, Harvard University.
E-mail: hlavac at fas.harvard.edu \% Date and time: ven., oct. 02, 2015
- 12:04:23

\begin{table}[!htbp] \centering 
  \caption{Orientation des AVC} 
  \label{orientation} 
\begin{tabular}{@{\extracolsep{5pt}} cccccccccc} 
\\[-1.8ex]\hline 
\hline \\[-1.8ex] 
CHIR & FUGUE & HO & MED & REA & SC & SCAM & SI & UHCD & NA's \\ 
\hline \\[-1.8ex] 
$75$ & $1$ & $1$ & $720$ & $68$ & $46$ & $9$ & $361$ & $919$ & $749$ \\ 
\hline \\[-1.8ex] 
\end{tabular} 
\end{table}

\section{Analyse par type
d'étblissement}\label{analyse-par-type-detblissement}

Voir routine \textbf{analyse-type\_etablissement} (rapport\_2014.R).

\subsection{SU de CHU}\label{su-de-chu}

Un seul établissement \textbf{HUS} avec 3 SU:

\begin{itemize}
\itemsep1pt\parskip0pt\parsep0pt
\item
  NHC
\item
  HTP Adultes
\item
  HTP Pédiatrie
\end{itemize}

\begin{Shaded}
\begin{Highlighting}[]
\NormalTok{es <-}\StringTok{ }\NormalTok{dx[dx$FINESS ==}\StringTok{ "Hus"}\NormalTok{,]}

\CommentTok{# à gerder ou supprimer mais redondant}
 \NormalTok{a <-}\StringTok{ }\KeywordTok{analyse_type_etablissement}\NormalTok{(es)}

    \CommentTok{# nombre de passages déclarés}
    \NormalTok{n.passages <-}\StringTok{ }\KeywordTok{nrow}\NormalTok{(es)}

    \CommentTok{# Nombre de RPU avec un âge renseigné}
    \NormalTok{age <-}\StringTok{ }\KeywordTok{summary.age}\NormalTok{(es$AGE)}

    \CommentTok{# Nombre de RPU avec un code postal renseigné}
    \NormalTok{cp <-}\StringTok{ }\KeywordTok{summary.cp}\NormalTok{(es$CODE_POSTAL)}
    
    \CommentTok{# passages de nuit}
    \NormalTok{nuit <-}\StringTok{ }\KeywordTok{passage}\NormalTok{(}\KeywordTok{horaire}\NormalTok{(es$ENTREE), }\StringTok{"nuit"}\NormalTok{)}

    \CommentTok{# passage en PDS}
    \NormalTok{pds <-}\StringTok{ }\KeywordTok{table}\NormalTok{(}\KeywordTok{pdsa}\NormalTok{(es$ENTREE))}

    \CommentTok{#Nombre de RPU avec une date et heure d'entrée renseignées}
    \NormalTok{entree <-}\StringTok{ }\KeywordTok{summary.dateheure}\NormalTok{(es$ENTREE)}
    \NormalTok{sortie <-}\StringTok{ }\KeywordTok{summary.dateheure}\NormalTok{(es$SORTIE)}

    \CommentTok{# nombre avec moyen de transport renseigné}
    \NormalTok{s <-}\StringTok{ }\KeywordTok{summary.transport}\NormalTok{(es$TRANSPORT)}
    \NormalTok{sn <-}\StringTok{ }\KeywordTok{c}\NormalTok{(s[}\StringTok{'n.na'}\NormalTok{], s[}\StringTok{'n.rens'}\NormalTok{], s[}\StringTok{'n.perso'}\NormalTok{], s[}\StringTok{'n.ambu'}\NormalTok{], s[}\StringTok{'n.vsav'}\NormalTok{], s[}\StringTok{'n.smur'}\NormalTok{], s[}\StringTok{'n.heli'}\NormalTok{], s[}\StringTok{'n.fo'}\NormalTok{])}
    \NormalTok{sp <-}\StringTok{ }\KeywordTok{c}\NormalTok{(s[}\StringTok{'p.na'}\NormalTok{], s[}\StringTok{'p.rens'}\NormalTok{], s[}\StringTok{'p.perso'}\NormalTok{], s[}\StringTok{'p.ambu'}\NormalTok{], s[}\StringTok{'p.vsav'}\NormalTok{], s[}\StringTok{'p.smur'}\NormalTok{], s[}\StringTok{'p.heli'}\NormalTok{], s[}\StringTok{'p.fo'}\NormalTok{])}
    \NormalTok{r <-}\StringTok{ }\KeywordTok{rbind}\NormalTok{(sn, }\KeywordTok{round}\NormalTok{(sp*}\DecValTok{100}\NormalTok{,}\DecValTok{2}\NormalTok{))}
    \NormalTok{r <-}\StringTok{ }\KeywordTok{t}\NormalTok{(r)}
    \KeywordTok{colnames}\NormalTok{(r) <-}\StringTok{ }\KeywordTok{c}\NormalTok{(}\StringTok{"n"}\NormalTok{, }\StringTok{"%"}\NormalTok{)}
    \KeywordTok{rownames}\NormalTok{(r) <-}\StringTok{ }\KeywordTok{c}\NormalTok{(}\StringTok{"Renseignés"}\NormalTok{,}\StringTok{"Non renseignés"}\NormalTok{,}\StringTok{"Moyen personnel"}\NormalTok{,}\StringTok{"Ambulance"}\NormalTok{,}\StringTok{"VSAV"}\NormalTok{,}\StringTok{"SMUR"}\NormalTok{,}\StringTok{"Hélicoptère"}\NormalTok{, }\StringTok{"Forces de l'ordre"}\NormalTok{)}
    \NormalTok{transport <-}\StringTok{ }\NormalTok{r}

    

    \CommentTok{# nombre avec CCMU renseigné}
    \KeywordTok{summary.ccmu}\NormalTok{(es$GRAVITE)}
\end{Highlighting}
\end{Shaded}

\begin{verbatim}
##         n      n.na      p.na    n.rens    p.rens   n.ccmu1   n.ccmu2 
## 61793.000 28298.000     0.458 33495.000     0.542  8743.000 17870.000 
##   n.ccmu3   n.ccmu4   n.ccmu5   n.ccmup   n.ccmud   p.ccmu1   p.ccmu2 
##  6178.000   503.000   201.000        NA        NA     0.261     0.534 
##   p.ccmu3   p.ccmu4   p.ccmu5   p.ccmup   p.ccmud 
##     0.184     0.015     0.006        NA        NA
\end{verbatim}

\begin{Shaded}
\begin{Highlighting}[]
    \CommentTok{# nombre de sorties conformes}
    \KeywordTok{summary.passages}\NormalTok{(}\KeywordTok{duree.passage2}\NormalTok{(es))}
\end{Highlighting}
\end{Shaded}

\begin{verbatim}
##                 n.conforme      duree.moyenne.passage 
##                      26416                        254 
##      duree.mediane.passage  duree.moyenne.passage.dom 
##                        141                        827 
##  duree.mediane.passage.dom duree.moyenne.passage.hosp 
##                        664                        346 
## duree.mediane.passage.hosp                 n.passage4 
##                        312                      17899 
##            n.hosp.passage4             n.dom.passage4 
##                       1225                      16674 
##                      n.dom                     n.hosp 
##                       2958                       3320 
##                n.transfert                 n.mutation 
##                        115                       3205 
##                    n.deces 
##                          0
\end{verbatim}

\begin{Shaded}
\begin{Highlighting}[]
    \CommentTok{# Nombre de RPU avec un mode de sortie renseigné}
    \KeywordTok{summary.mode.sortie}\NormalTok{(es$MODE_SORTIE)}
\end{Highlighting}
\end{Shaded}

\begin{verbatim}
##           n        n.na        p.na      n.rens      p.rens       n.dom 
##  61793.0000  35337.0000      0.5719  26456.0000      0.4281   3122.0000 
##      n.hosp n.transfert  n.mutation     n.deces       p.dom      p.hosp 
##  23334.0000    115.0000  23219.0000      0.0000      0.1180      0.8820 
## p.transfert  p.mutation     p.deces 
##      0.0043      0.8776      0.0000
\end{verbatim}

\begin{itemize}
\itemsep1pt\parskip0pt\parsep0pt
\item
  Nombre de passages déclarés: 61 793 en 2014.
\item
  Nombre de RPU avec un âge renseigné: 61 793.
\item
  Nombre de RPU avec un code postal renseigné: 61 793.
\item
  Nombre de passages par jour de la semaine:
\end{itemize}

\begin{table}[ht]
\centering
\begin{tabular}{rlll}
  \hline
 & Jour & n & \% \\ 
  \hline
1 & Lun & 9 211 & 14.91 \\ 
  2 & Mar & 8 980 & 14.53 \\ 
  3 & Mer & 8 527 & 13.8 \\ 
  4 & Jeu & 8 667 & 14.03 \\ 
  5 & Ven & 9 170 & 14.84 \\ 
  6 & Sam & 8 806 & 14.25 \\ 
  7 & Dim & 8 432 & 13.65 \\ 
   \hline
\end{tabular}
\caption{Nombre de RPU par jour de semaine} 
\end{table}

\begin{itemize}
\itemsep1pt\parskip0pt\parsep0pt
\item
  Nombre d'âges renseignés: 61 793
\end{itemize}

\begin{table}[ht]
\centering
\begin{tabular}{rlll}
  \hline
 & Tranches d'âge & n & \% \\ 
  \hline
Moins de 1 an & n.inf1an &  2 888 & 9.72 \\ 
  Moins de 15 ans & n.inf15ans & 14 472 & 48.73 \\ 
  75 ans et plus & n.75ans & 12 337 & 41.54 \\ 
   \hline
\end{tabular}
\caption{Nombre de RPU par tranches d'âge} 
\end{table}

\begin{itemize}
\item
  Nombre de passages la nuit: 22 681 (36.7 \%)
\item
  Nombre de passages en horaire de PDS:

  \begin{table}[ht]
  \centering
  \begin{tabular}{rlll}
    \hline
   & Jour & n & \% \\ 
    \hline
  1 & NPDS & 30 243 & 48.94 \\ 
    2 & PDSS & 16 016 & 25.92 \\ 
    3 & PDSWE & 15 534 & 25.14 \\ 
     \hline
  \end{tabular}
  \caption{Passages aux urgences aux horaires de permanence des soins: Passages totaux (NPDS), passages  en semaine (PDSS), passages le week-end (PDSWE)} 
  \end{table}
\item
  Nombre de RPU avec une date-heure d'entrée renseignés: 61 793 (100 \%)
\item
  Nombre de RPU avec une date-heure de sortie renseignés: 48 631 (78.7
  \%)
\item
  Moyen de transport utilisé pour se rendre aux urgences:

  \begin{table}[ht]
  \centering
  \begin{tabular}{rrr}
    \hline
   & n & \% \\ 
    \hline
  Renseignés & 53 476,00 & 86,54 \\ 
    Non renseignés & 8 317,00 & 13,46 \\ 
    Moyen personnel & 1 214,00 & 14,60 \\ 
    Ambulance & 4 518,00 & 54,32 \\ 
    VSAV & 2 262,00 & 27,20 \\ 
    SMUR & 274,00 & 3,29 \\ 
    Hélicoptère & 2,00 & 0,02 \\ 
    Forces de l'ordre & 47,00 & 0,57 \\ 
     \hline
  \end{tabular}
  \caption{Moyens de transport} 
  \end{table}
\end{itemize}

\subsection{SU d'ES siège de SAMU, non
CHU}\label{su-des-siege-de-samu-non-chu}

Un seul établissement: CH de Mulhouse avec 2 implantations:

\begin{itemize}
\itemsep1pt\parskip0pt\parsep0pt
\item
  Emile Muller (Adultes + Pédiatrie traumatique)
\item
  Hasenrain (Pédiatrie médicale)
\end{itemize}

\begin{Shaded}
\begin{Highlighting}[]
\NormalTok{es <-}\StringTok{ }\NormalTok{dx[dx$FINESS ==}\StringTok{ "Mul"}\NormalTok{,]}
\CommentTok{# analyse_type_etablissement(es)}

    \CommentTok{# nombre de passages déclarés}
    \KeywordTok{nrow}\NormalTok{(es)}
\end{Highlighting}
\end{Shaded}

\begin{verbatim}
[1] 59471
\end{verbatim}

\begin{Shaded}
\begin{Highlighting}[]
    \CommentTok{# Nombre de RPU avec un âge renseigné}
    \KeywordTok{summary.age}\NormalTok{(es$AGE)}
\end{Highlighting}
\end{Shaded}

\begin{verbatim}
         n       n.na       p.na     n.rens     p.rens   n.inf1an 
 59471.000      0.000      0.000  59471.000      1.000   4715.000 
n.inf15ans n.inf18ans    n.75ans    n.85ans    n.90ans   p.inf1an 
 19015.000  20847.000   7480.000   3152.000   1226.000      0.079 
p.inf15ans p.inf18ans    p.75ans    p.85ans    p.90ans   mean.age 
     0.320      0.351      0.126      0.053      0.021     34.600 
    sd.age median.age    min.age    max.age         q1         q3 
    28.078     30.000      0.000    113.000      8.000     57.000 
\end{verbatim}

\begin{Shaded}
\begin{Highlighting}[]
    \CommentTok{# Nombre de RPU avec un code postal renseigné}
    \KeywordTok{summary.cp}\NormalTok{(es$CODE_POSTAL)}
\end{Highlighting}
\end{Shaded}

\begin{verbatim}
          n        n.na        p.na      n.rens      p.rens n.residents 
      59471           0           0       59471           1       57952 
n.etrangers 
       1519 
\end{verbatim}

\begin{Shaded}
\begin{Highlighting}[]
    \CommentTok{# par jour de semaine}
    \KeywordTok{summary.wday}\NormalTok{(es$ENTREE)}
\end{Highlighting}
\end{Shaded}

\begin{verbatim}
 Lun  Mar  Mer  Jeu  Ven  Sam  Dim 
8868 7885 8130 7931 8270 8854 9533 
\end{verbatim}

\begin{Shaded}
\begin{Highlighting}[]
    \KeywordTok{summary.age}\NormalTok{(es$AGE)}
\end{Highlighting}
\end{Shaded}

\begin{verbatim}
         n       n.na       p.na     n.rens     p.rens   n.inf1an 
 59471.000      0.000      0.000  59471.000      1.000   4715.000 
n.inf15ans n.inf18ans    n.75ans    n.85ans    n.90ans   p.inf1an 
 19015.000  20847.000   7480.000   3152.000   1226.000      0.079 
p.inf15ans p.inf18ans    p.75ans    p.85ans    p.90ans   mean.age 
     0.320      0.351      0.126      0.053      0.021     34.600 
    sd.age median.age    min.age    max.age         q1         q3 
    28.078     30.000      0.000    113.000      8.000     57.000 
\end{verbatim}

\begin{Shaded}
\begin{Highlighting}[]
    \CommentTok{# passages de nuit}
    \KeywordTok{passage}\NormalTok{(}\KeywordTok{horaire}\NormalTok{(es$ENTREE), }\StringTok{"nuit"}\NormalTok{)}
\end{Highlighting}
\end{Shaded}

\begin{verbatim}
[1] 18349.00     0.31
\end{verbatim}

\begin{Shaded}
\begin{Highlighting}[]
    \CommentTok{# passage en PDS}
    \KeywordTok{table}\NormalTok{(}\KeywordTok{pdsa}\NormalTok{(es$ENTREE))}
\end{Highlighting}
\end{Shaded}

\begin{verbatim}

 NPDS  PDSS PDSWE 
30530 12634 16307 
\end{verbatim}

\begin{Shaded}
\begin{Highlighting}[]
    \CommentTok{#Nombre de RPU avec une date et heure d'entrée renseignées}
    \KeywordTok{summary.dateheure}\NormalTok{(es$ENTREE)}
\end{Highlighting}
\end{Shaded}

\begin{verbatim}
     n   n.na   p.na n.rens p.rens 
 59471      0      0  59471      1 
\end{verbatim}

\begin{Shaded}
\begin{Highlighting}[]
    \CommentTok{# nombre avec moyen de transport renseigné}
    \KeywordTok{summary.transport}\NormalTok{(es$TRANSPORT)}
\end{Highlighting}
\end{Shaded}

\begin{verbatim}
         n       n.na       p.na     n.rens     p.rens       n.fo 
59471.0000  3836.0000     0.0645 55635.0000     0.9355   478.0000 
    n.heli    n.perso     n.smur     n.vsav     n.ambu       p.fo 
  122.0000 35973.0000   223.0000  7051.0000 11788.0000     0.0086 
    p.heli    p.perso     p.smur     p.vsav     p.ambu 
    0.0022     0.6466     0.0040     0.1267     0.2119 
\end{verbatim}

\begin{Shaded}
\begin{Highlighting}[]
    \CommentTok{# nombre avec CCMU renseigné}
    \KeywordTok{summary.ccmu}\NormalTok{(es$GRAVITE)}
\end{Highlighting}
\end{Shaded}

\begin{verbatim}
         n       n.na       p.na     n.rens     p.rens    n.ccmu1 
59471.0000 14043.0000     0.2361 45428.0000     0.7639  7349.0000 
   n.ccmu2    n.ccmu3    n.ccmu4    n.ccmu5    n.ccmup    n.ccmud 
30451.0000  6094.0000  1235.0000   299.0000         NA         NA 
   p.ccmu1    p.ccmu2    p.ccmu3    p.ccmu4    p.ccmu5    p.ccmup 
    0.1618     0.6703     0.1341     0.0272     0.0066         NA 
   p.ccmud 
        NA 
\end{verbatim}

\begin{Shaded}
\begin{Highlighting}[]
    \CommentTok{# nombre de sorties conformes}
    \KeywordTok{summary.passages}\NormalTok{(}\KeywordTok{duree.passage2}\NormalTok{(es))}
\end{Highlighting}
\end{Shaded}

\begin{verbatim}
                n.conforme      duree.moyenne.passage 
                     47518                        184 
     duree.mediane.passage  duree.moyenne.passage.dom 
                       151                        172 
 duree.mediane.passage.dom duree.moyenne.passage.hosp 
                       141                        244 
duree.mediane.passage.hosp                 n.passage4 
                       219                      34593 
           n.hosp.passage4             n.dom.passage4 
                      4159                      30434 
                     n.dom                     n.hosp 
                     35746                       7543 
               n.transfert                 n.mutation 
                       135                       7408 
                   n.deces 
                         0 
\end{verbatim}

\begin{Shaded}
\begin{Highlighting}[]
    \CommentTok{# Nombre de RPU avec un mode de sortie renseigné}
    \KeywordTok{summary.mode.sortie}\NormalTok{(es$MODE_SORTIE)}
\end{Highlighting}
\end{Shaded}

\begin{verbatim}
          n        n.na        p.na      n.rens      p.rens       n.dom 
 59471.0000  14257.0000      0.2397  45214.0000      0.7603  36717.0000 
     n.hosp n.transfert  n.mutation     n.deces       p.dom      p.hosp 
  8497.0000    150.0000   8347.0000      0.0000      0.8121      0.1879 
p.transfert  p.mutation     p.deces 
     0.0033      0.1846      0.0000 
\end{verbatim}

\subsection{SU avec SMUR non siège de
SAMU}\label{su-avec-smur-non-siege-de-samu}

SU abec SMUR sans SAMU, 5 établissements:

\begin{itemize}
\itemsep1pt\parskip0pt\parsep0pt
\item
  CH Wissembourg
\item
  CH haguenau
\item
  CH Saverne
\item
  CH Sélestat
\item
  CH Colmar
\end{itemize}

\begin{Shaded}
\begin{Highlighting}[]
\NormalTok{es <-}\StringTok{ }\NormalTok{dx[dx$FINESS %in%}\StringTok{ }\KeywordTok{c}\NormalTok{(}\StringTok{"Wis"}\NormalTok{,}\StringTok{"Hag"}\NormalTok{,}\StringTok{"Sav"}\NormalTok{,}\StringTok{"Sel"}\NormalTok{,}\StringTok{"Col"}\NormalTok{),]}
\CommentTok{# analyse_type_etablissement(es)}

\CommentTok{# nombre de passages déclarés}
    \KeywordTok{nrow}\NormalTok{(es)}
\end{Highlighting}
\end{Shaded}

\begin{verbatim}
[1] 177747
\end{verbatim}

\begin{Shaded}
\begin{Highlighting}[]
    \CommentTok{# Nombre de RPU avec un âge renseigné}
    \KeywordTok{summary.age}\NormalTok{(es$AGE)}
\end{Highlighting}
\end{Shaded}

\begin{verbatim}
         n       n.na       p.na     n.rens     p.rens   n.inf1an 
177747.000      0.000      0.000 177747.000      1.000   7110.000 
n.inf15ans n.inf18ans    n.75ans    n.85ans    n.90ans   p.inf1an 
 48795.000  55583.000  25444.000  10810.000   4308.000      0.040 
p.inf15ans p.inf18ans    p.75ans    p.85ans    p.90ans   mean.age 
     0.275      0.313      0.143      0.061      0.024     37.300 
    sd.age median.age    min.age    max.age         q1         q3 
    27.737     33.000      0.000    120.000     13.000     59.000 
\end{verbatim}

\begin{Shaded}
\begin{Highlighting}[]
    \CommentTok{# Nombre de RPU avec un code postal renseigné}
    \KeywordTok{summary.cp}\NormalTok{(es$CODE_POSTAL)}
\end{Highlighting}
\end{Shaded}

\begin{verbatim}
          n        n.na        p.na      n.rens      p.rens n.residents 
     177747           0           0      177747           1      166676 
n.etrangers 
      11071 
\end{verbatim}

\begin{Shaded}
\begin{Highlighting}[]
    \CommentTok{# par jour de semaine}
    \KeywordTok{summary.wday}\NormalTok{(es$ENTREE)}
\end{Highlighting}
\end{Shaded}

\begin{verbatim}
  Lun   Mar   Mer   Jeu   Ven   Sam   Dim 
27415 24007 24628 24099 24688 25896 27014 
\end{verbatim}

\begin{Shaded}
\begin{Highlighting}[]
    \KeywordTok{summary.age}\NormalTok{(es$AGE)}
\end{Highlighting}
\end{Shaded}

\begin{verbatim}
         n       n.na       p.na     n.rens     p.rens   n.inf1an 
177747.000      0.000      0.000 177747.000      1.000   7110.000 
n.inf15ans n.inf18ans    n.75ans    n.85ans    n.90ans   p.inf1an 
 48795.000  55583.000  25444.000  10810.000   4308.000      0.040 
p.inf15ans p.inf18ans    p.75ans    p.85ans    p.90ans   mean.age 
     0.275      0.313      0.143      0.061      0.024     37.300 
    sd.age median.age    min.age    max.age         q1         q3 
    27.737     33.000      0.000    120.000     13.000     59.000 
\end{verbatim}

\begin{Shaded}
\begin{Highlighting}[]
    \CommentTok{# passages de nuit}
    \KeywordTok{passage}\NormalTok{(}\KeywordTok{horaire}\NormalTok{(es$ENTREE), }\StringTok{"nuit"}\NormalTok{)}
\end{Highlighting}
\end{Shaded}

\begin{verbatim}
[1] 46677.00     0.26
\end{verbatim}

\begin{Shaded}
\begin{Highlighting}[]
    \CommentTok{# passage en PDS}
    \KeywordTok{table}\NormalTok{(}\KeywordTok{pdsa}\NormalTok{(es$ENTREE))}
\end{Highlighting}
\end{Shaded}

\begin{verbatim}

 NPDS  PDSS PDSWE 
98847 32166 46734 
\end{verbatim}

\begin{Shaded}
\begin{Highlighting}[]
    \CommentTok{#Nombre de RPU avec une date et heure d'entrée renseignées}
    \KeywordTok{summary.dateheure}\NormalTok{(es$ENTREE)}
\end{Highlighting}
\end{Shaded}

\begin{verbatim}
     n   n.na   p.na n.rens p.rens 
177747      0      0 177747      1 
\end{verbatim}

\begin{Shaded}
\begin{Highlighting}[]
    \CommentTok{# nombre avec moyen de transport renseigné}
    \KeywordTok{summary.transport}\NormalTok{(es$TRANSPORT)}
\end{Highlighting}
\end{Shaded}

\begin{verbatim}
          n        n.na        p.na      n.rens      p.rens        n.fo 
177747.0000  41213.0000      0.2319 136534.0000      0.7681    760.0000 
     n.heli     n.perso      n.smur      n.vsav      n.ambu        p.fo 
    95.0000  97489.0000   1603.0000  14816.0000  21771.0000      0.0056 
     p.heli     p.perso      p.smur      p.vsav      p.ambu 
     0.0007      0.7140      0.0117      0.1085      0.1595 
\end{verbatim}

\begin{Shaded}
\begin{Highlighting}[]
    \CommentTok{# nombre avec CCMU renseigné}
    \KeywordTok{summary.ccmu}\NormalTok{(es$GRAVITE)}
\end{Highlighting}
\end{Shaded}

\begin{verbatim}
           n         n.na         p.na       n.rens       p.rens 
177747.00000  21649.00000      0.12180 156098.00000      0.87820 
     n.ccmu1      n.ccmu2      n.ccmu3      n.ccmu4      n.ccmu5 
 27108.00000 101455.00000  24309.00000   1547.00000    385.00000 
     n.ccmup      n.ccmud      p.ccmu1      p.ccmu2      p.ccmu3 
  1273.00000     21.00000      0.17366      0.64994      0.15573 
     p.ccmu4      p.ccmu5      p.ccmup      p.ccmud 
     0.00991      0.00247      0.00816      0.00013 
\end{verbatim}

\begin{Shaded}
\begin{Highlighting}[]
    \CommentTok{# nombre de sorties conformes}
    \KeywordTok{summary.passages}\NormalTok{(}\KeywordTok{duree.passage2}\NormalTok{(es))}
\end{Highlighting}
\end{Shaded}

\begin{verbatim}
                n.conforme      duree.moyenne.passage 
                    158099                        164 
     duree.mediane.passage  duree.moyenne.passage.dom 
                       120                        144 
 duree.mediane.passage.dom duree.moyenne.passage.hosp 
                       107                        245 
duree.mediane.passage.hosp                 n.passage4 
                       206                     126600 
           n.hosp.passage4             n.dom.passage4 
                     18044                     108555 
                     n.dom                     n.hosp 
                    125505                      30709 
               n.transfert                 n.mutation 
                      2840                      27869 
                   n.deces 
                         1 
\end{verbatim}

\begin{Shaded}
\begin{Highlighting}[]
    \CommentTok{# Nombre de RPU avec un mode de sortie renseigné}
    \KeywordTok{summary.mode.sortie}\NormalTok{(es$MODE_SORTIE)}
\end{Highlighting}
\end{Shaded}

\begin{verbatim}
            n          n.na          p.na        n.rens        p.rens 
177747.000000  10215.000000      0.057469 167532.000000      0.942531 
        n.dom        n.hosp   n.transfert    n.mutation       n.deces 
126977.000000  40554.000000   2990.000000  37564.000000      1.000000 
        p.dom        p.hosp   p.transfert    p.mutation       p.deces 
     0.757927      0.242067      0.017847      0.224220      0.000006 
\end{verbatim}

\subsection{SU non SMUR, non SAMU, non
CHU}\label{su-non-smur-non-samu-non-chu}

ES avec SU isolé (pas de SMUR associé): 8 établissements

\begin{itemize}
\itemsep1pt\parskip0pt\parsep0pt
\item
  Ste Anne
\item
  Ste Odile
\item
  Diaconat Strasbourg
\item
  CH de Guebwiller
\item
  CH de Thann (pas de RPU)
\item
  CH d'Altkirch
\item
  Clinique des 3 frontières
\item
  Roosvelt
\item
  Fonderie
\end{itemize}

\begin{verbatim}
     n.passages       n.age.ren        n.inf1an      n.inf15ans 
         117722          117718             663           21131 
        n.75ans       n.cp.rens     n.etrangers           n.lun 
          12010          117722            3372           18645 
          n.mar           n.mer           n.jeu           n.ven 
          15880           16232           16002           16456 
          n.sam           n.dim          n.nuit           n.pds 
          17567           16940           27711           49063 
       n.h.rens    n.trans.rens            n.fo          n.heli 
         117722           88822             264               1 
        n.perso          n.smur          n.vsav          n.ambu 
          74095             602            5825            8035 
    n.ccmu.rens         n.ccmu1         n.ccmu2         n.ccmu3 
         104806            8482           85521           10593 
        n.ccmu4         n.ccmu5         n.ccmuP         n.ccmuD 
            147              24              34               5 
       n.ccmu45  n.sorties.conf    mean.passage  median.passage 
            171          106689             120              87 
     n.passage4 n.hosp.passage4  n.dom.passage4           n.dom 
          96529            7050           89478           86531 
         n.hosp     n.transfert         n.deces   n.mode.sortie 
           9031            2575               1           99676 
    n.mutation2 
           7869 
\end{verbatim}

Test de la routine et tableau compact

\begin{verbatim}
##                 es.chu es.samu es.smur es.simple
## n.passages       61793   59471  177747    117722
## n.age.ren        61793   59471  177747    117718
## n.inf1an          2888    4715    7110       663
## n.inf15ans       14472   19015   48795     21131
## n.75ans          12337    7480   25444     12010
## n.cp.rens        61793   59471  177747    117722
## n.etrangers       2505    1519   11071      3372
## n.lun             9211    8868   27415     18645
## n.mar             8980    7885   24007     15880
## n.mer             8527    8130   24628     16232
## n.jeu             8667    7931   24099     16002
## n.ven             9170    8270   24688     16456
## n.sam             8806    8854   25896     17567
## n.dim             8432    9533   27014     16940
## n.nuit           22681   18349   46677     27711
## n.pds            31550   28941   78900     49063
## n.h.rens         61793   59471  177747    117722
## n.trans.rens      8317   55635  136534     88822
## n.fo                47     478     760       264
## n.heli               2     122      95         1
## n.perso           1214   35973   97489     74095
## n.smur             274     223    1603       602
## n.vsav            2262    7051   14816      5825
## n.ambu            4518   11788   21771      8035
## n.ccmu.rens      33495   45428  156098    104806
## n.ccmu1           8743    7349   27108      8482
## n.ccmu2          17870   30451  101455     85521
## n.ccmu3           6178    6094   24309     10593
## n.ccmu4            503    1235    1547       147
## n.ccmu5            201     299     385        24
## n.ccmuP             NA      NA    1273        34
## n.ccmuD             NA      NA      21         5
## n.ccmu45           704    1534    1932       171
## n.sorties.conf   26416   47518  158099    106689
## mean.passage       254     184     164       120
## median.passage     141     151     120        87
## n.passage4       17899   34593  126600     96529
## n.hosp.passage4   1225    4159   18044      7050
## n.dom.passage4   16674   30434  108555     89478
## n.dom             2958   35746  125505     86531
## n.hosp            3320    7543   30709      9031
## n.transfert        115     135    2840      2575
## n.deces              0       0       1         1
## n.mode.sortie    26456   45214  167532     99676
## n.mutation2      23219    8347   37564      7869
\end{verbatim}

\subsection{Doublons ?}\label{doublons}

\begin{itemize}
\itemsep1pt\parskip0pt\parsep0pt
\item
  Age moyen, année N
\item
  Répartition par classe âge en pourcentage, année N
\item
  Répartition par sexe en pourcentage, année N
\item
  TOP 5 pourcentage par code CIM 10, année N
\item
  Répartition we/semaine en pourcentage, année N
\item
  Répartition par tranche heure en pourcentage, année N
\end{itemize}

\subsection{Comparaison à hôpitaux
constants}\label{comparaison-a-hopitaux-constants}

Introduit dans le rapoort 2014: pour évaluer le taux de progression
2013-2014, on compare uniquement les hôpitaux de 2014 qui transmettaient
déjà des données en 2013:

\begin{verbatim}
## [1] 0.17
\end{verbatim}

\begin{verbatim}
## [1] 115051
\end{verbatim}

\begin{verbatim}
## [1] 56061
\end{verbatim}

\section{ANNEXES}\label{annexes}

\subsection{ANNEXE 1 : Définitions}\label{annexe-1-definitions}

\subsection{ANNEXE 2 : Diagramme de complétude des
RPU}\label{annexe-2-diagramme-de-completude-des-rpu}

\subsection{ANNEXE 3 : Calcul du TARRU}\label{annexe-3-calcul-du-tarru}

\section{Information de session}\label{information-de-session}

\begin{verbatim}
R version 3.2.2 (2015-08-14)
Platform: x86_64-pc-linux-gnu (64-bit)
Running under: Ubuntu 14.04.3 LTS

locale:
 [1] LC_CTYPE=fr_FR.UTF-8       LC_NUMERIC=C              
 [3] LC_TIME=fr_FR.UTF-8        LC_COLLATE=fr_FR.UTF-8    
 [5] LC_MONETARY=fr_FR.UTF-8    LC_MESSAGES=fr_FR.UTF-8   
 [7] LC_PAPER=fr_FR.UTF-8       LC_NAME=C                 
 [9] LC_ADDRESS=C               LC_TELEPHONE=C            
[11] LC_MEASUREMENT=fr_FR.UTF-8 LC_IDENTIFICATION=C       

attached base packages:
[1] stats     graphics  grDevices utils     datasets  methods   base     

other attached packages:
 [1] openintro_1.4     xtable_1.7-4      stargazer_5.2    
 [4] epicalc_2.15.1.0  nnet_7.3-11       MASS_7.3-44      
 [7] survival_2.38-3   foreign_0.8-66    R.utils_2.1.0    
[10] R.oo_1.19.0       R.methodsS3_1.7.0 xts_0.9-7        
[13] zoo_1.7-12        plotrix_3.5-12    lubridate_1.3.3  
[16] knitr_1.11       

loaded via a namespace (and not attached):
 [1] Rcpp_0.12.1     magrittr_1.5    splines_3.2.2   lattice_0.20-33
 [5] highr_0.5.1     stringr_1.0.0   plyr_1.8.3      tools_3.2.2    
 [9] grid_3.2.2      htmltools_0.2.6 yaml_2.1.13     digest_0.6.8   
[13] formatR_1.2.1   memoise_0.2.1   evaluate_0.8    rmarkdown_0.8  
[17] stringi_0.5-5  
\end{verbatim}

\begin{verbatim}

To cite R in publications use:

  R Core Team (2015). R: A language and environment for
  statistical computing. R Foundation for Statistical Computing,
  Vienna, Austria. URL https://www.R-project.org/.

A BibTeX entry for LaTeX users is

  @Manual{,
    title = {R: A Language and Environment for Statistical Computing},
    author = {{R Core Team}},
    organization = {R Foundation for Statistical Computing},
    address = {Vienna, Austria},
    year = {2015},
    url = {https://www.R-project.org/},
  }

We have invested a lot of time and effort in creating R, please
cite it when using it for data analysis. See also
'citation("pkgname")' for citing R packages.
\end{verbatim}

\section{Temps de calcul}\label{temps-de-calcul}

\begin{verbatim}
##    user  system elapsed 
##    81.3     2.4    83.7
\end{verbatim}

\end{document}
